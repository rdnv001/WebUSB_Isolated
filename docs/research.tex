%%%%%%%%%%%% Attribution %%%%%%%%%%%%
% This template was created by 
% Chuck F. Rocca at WCSU and may be
% copied and used freely for 
% non-commercial purposes.
% 10-17-2021
%%%%%%%%%%%%%%%%%%%%%%%%%%%%%%%%%%%%%

%%%%%%% Start Document Header %%%%%%%
% In creating a new document
% copy and paste the header 
% as is.
%%%%%%%%%%%%%%%%%%%%%%%%%%%%%%%%%%%%%

\documentclass[12pt]{article}
\usepackage[russian]{babel}
\usepackage{fancyvrb,minted,tcolorbox}
\usepackage[T1]{fontenc}
\usepackage{textcomp}
\usepackage{listings}
\usepackage{hyperref}
\usepackage[scale=0.98,ttdefault]{AnonymousPro}
\tcbuselibrary{skins,breakable,breakable}
\tcbuselibrary{minted}
\definecolor{terminalColor}{RGB}{16,16,16}
\definecolor{Button1}{RGB}{254,94,86}
\definecolor{Button2}{RGB}{254,188,45}
\definecolor{Button3}{RGB}{38,202,59}

\usepackage[left=1.5cm,right=1.5cm,top=1.5cm,bottom=1.5cm,ignoreheadfoot]{geometry}
\usepackage{array}
\usepackage[svgnames,table]{xcolor}

\usepackage{tcolorbox}
\tcbuselibrary{theorems}

\usepackage{listings}
\usepackage{xcolor}


\definecolor{codegreen}{rgb}{0,0.6,0}
\definecolor{codegray}{rgb}{0.5,0.5,0.5}
\definecolor{codepurple}{rgb}{0.58,0,0.82}
\definecolor{backcolour}{rgb}{0.95,0.95,0.92}

\lstdefinestyle{c@mmon}{%
extendedchars=true,
showtabs=true,
tab=    ,
tabsize=4,
basicstyle=\selectlanguage{russian}\ttfamily\footnotesize,
breaklines=true,
breakatwhitespace=true,
aboveskip=2ex,
frame=trbl,
rulecolor=\color{black},
backgroundcolor=\color{yellow!10},
xleftmargin=5pt,
xrightmargin=5pt,
numbers=left
}
\lstset{style=c@mmon}

\lstdefinestyle{mystyle}{
language={[ANSI]C++},
style=c@mmon,
stringstyle=\frenchspacing\color{stringcolour},
showstringspaces=false,
alsoletter={1234567890},
keywordstyle=\color{keywordcolour}\bfseries,
emph={auto,const,struct,%
break,continue,else,for,switch,void,%
case,default,enum,goto,register,sizeof,typedef,%
char,do,extern,if,return,static,union,while,%
asm,dynamic_cast,namespace,reinterpret_cast,try,%
bool,explicit,new,static_cast,typeid,volatile,%
catch,operator,template,typename,%
class,friend,private,this,using,%
const_cast,inline,public,throw,virtual,%
delete,mutable,protected,wchar_t,%
or,and,xor,not,assert},
emphstyle=\color{blue}\bfseries,
emph={[2]true, false, NULL},
emphstyle=[2]\color{keywordcolour},
emph={[3]double,float,int,short,unsigned,long,signed},
emphstyle=[3]\color{blue},
emph={[4]1, 2, 3, 4, 5, 6, 7, 8, 9, 0},
emph={[4]cos,sin,tan,acos,asin,atan,atan2,cosh,sinh,tanh,%
exp,frexp,ldexp,log,log10,modf,pow,sqrt,ceil,fabs,floor,fmod},
emphstyle=[4]\color{numpycolour},
literate=*%
{:}{{\literatecolour:}}{1}%
{=}{{\literatecolour=}}{1}%
{-}{{\literatecolour-}}{1}%
{+}{{\literatecolour+}}{1}%
{*}{{\literatecolour*}}{1}%
{!}{{\literatecolour!}}{1}%
{[}{{\literatecolour[}}{1}%
{]}{{\literatecolour]}}{1}%
{<}{{\literatecolour<}}{1}%
{>}{{\literatecolour>}}{1}%
{>>>}{{\textcolor{promptcolour}{>>>}}}{1}%
,%
commentstyle=\frenchspacing\color{green},
morecomment=[l][commentstyle]{//},
morecomment=[s][commentstyle]{/*}{*/},
morestring=[b][stringstyle]",
morestring=[d][stringstyle]',
}
\lstset{style=mystyle}

\newcommand*{\arraycolor}[1]{\protect\leavevmode\color{#1}}
\newcolumntype{A}{>{\columncolor{blue!50!white}}c}
\newcolumntype{B}{>{\columncolor{LightGoldenrod}}c}
\newcolumntype{C}{>{\columncolor{FireBrick!50}}c}
\newcolumntype{D}{>{\columncolor{Gray!42}}c}

\definecolor{gray}{gray}{0.5}
\colorlet{commentcolour}{green!50!black}
\colorlet{stringcolour}{red!60!black}
\colorlet{keywordcolour}{magenta!90!black}
\colorlet{exceptioncolour}{yellow!50!red}
\colorlet{commandcolour}{blue!60!black}
\colorlet{numpycolour}{blue!60!green}
\colorlet{literatecolour}{magenta!90!black}
\colorlet{promptcolour}{green!50!black}
\colorlet{specmethodcolour}{violet}
\colorlet{indendifiercolour}{green!70!white}

%%%% Header Information %%%%
    \input{header}

%%%% Document Information %%%%
    \title{Отчет по практике "Рутокен".}
    \author{Родионов Филипп}
    \date{25.07.24}

%%%%%%% End Document Header %%%%%%%


%%%% Begin Document %%%%
% note that the document starts with
% \begin{document} and ends with
% \end{document}
%%%%%%%%%%%%%%%%%%%%%%%%

\begin{document}

%%%% Format Running Header %%%%%
\markboth{\theauthor}{\thetitle}

%%%% Insert the Title Information %%%
\maketitle

%%%% Introduction to the General Template %%%%
\section{Формулировка вопроса}
\textbf{Выполнить обзор текущего состояния возможности доступа к смарт-картам из браузера через интерфейс WebUSB.}.
\section{Постановка задачи}
\textbf{Часть 1:} Разобраться, что такое Isolated Web App \\
\textbf{Часть 2:} Реализовать простейший Isolated Web App, использующий WebUSB API для доступа к Рутокен. \\

В рамках задания, в зависимости от статуса возможности доступа в описываемом окружении: предоставить исчерпывающие объяснения причин невозможности доступа или предоставить прототип, реализующий доступ к смарт-карте.

\begin{tcolorbox}
\textbf{Артефакты: }
\begin{enumerate}
    \item Отчет, описывающий, что такое Isolated Web App.
    \item PoC Isolated Web App с доступом к Рутокен. 
    \item Небольшая документация к PoC, поясняющая ключевые решения.
\end{enumerate}
\end{tcolorbox}

\section{Isolated Web App}
\textbf{Web USB} -- это JavaScript API, который даёт возможность веб-приложениям взаимодействовать с локальными USB-устройствами на компьютере. Согласно спецификации WebUSB, некоторые классы интерфейсов защищены от доступа веб-приложений, чтобы предотвратить доступ вредоносных скриптов к потенциально чувствительным данным. \\

Спецификация WebUSB определяет блок-лист уязвимых устройств и таблицу защищённых классов интерфейсов, доступ к которым заблокирован через WebUSB. С разрешением на использование функции «usb-unrestricted» изолированные веб приложения смогут получить доступ к устройствам из блок-листа и защищённым классам интерфейсов. \\
\textbf{Isolated Web App} -- это приложения, не размещённые на реальных веб-серверах, а упакованные в Web Bundles, подписанные их разработчиком и распространённые среди конечных пользователей. Обычно они создаются для внутреннего использования компаниями. 
Иными словами, веб приложение (сами файлы) хостится на компьютере конечного пользователя.\\

\section{Ход работы}
\subsection{WebUSB API внутри IWA}
Данное исследование направлено на изучение работы WebUSB API с устройствами, входящими в блок-лист устройств, с помощью изолированных веб приложений (IWA). \\
В качестве готового проекта, содержащего функции IWA, был взят \href{https://github.com/GoogleChromeLabs/telnet-client}{\textbf{telnet}}. 

Было обнаружено, что часть задачи, совпадающей с той, что была получена в начале уже выполнена в проекте \href{https://github.com/jbirkholz/webusbAuth}{\textbf{WebUSBAuth}}. \\
Проект демонстрирует взаимодействие со смарт-картой через WebUSB API при помощбю APDU комманд. Но, поскольку смарт карта входит в список запрещенных устройств, было предложено использовать архитектуру изолированного веб-приложения, реализованного в проекте \textbf{telnet}. 
Использование изолированного приложения позволяет обойти ограничения черного списка WebUSB API, предоставляя разработчикам доступ к функционалу смарт-карт. 

\subsection{WebUSBAuth тестовый запуск}
Было решено протестировать проект \textbf{WebUSBAuth} на MacOS. Далее - подробная инструкция по сборке и запуску проекта в Chrome 129. 
\begin{enumerate}
    \item Предварительно скачать архив проекта. 
    \item Предварительно скачать \href{https://www.python.org/downloads/macos/}{\textbf{python}}. 
    \item С помощью терминала перейти в директорию проекта. 
    \item Запустить python сервер. 
    \item В браузере перейти по адресу \emph{http://localhost:8000/demo.html}
\end{enumerate} 

При нажатии на кнопку connect reader и выборе нужного устройсва \emph{Рутокен} в панели выводится ошибка при вызове функции \emph{claimInterface}, которая происходит при попытке получения доступа к устройствам из \href{https://groups.google.com/a/chromium.org/g/blink-dev/c/LZXocaeCwDw/m/GLfAffGLAAAJ}{{блок-листа}} WebUSB API. \\
Для решения проблемы необходимо реализовать этот проект в изолированном приложении. \\

Но на данном этапе я еще не знал, что ошибка заключается именно в этом, потому решил «подсмотреть» как Linux общается с токеном с помощью специальных инструментов: \\
\begin{enumerate} 
    \item Для этого установить  Debian  (ARM64)  в  виртуальной машине UTM.
    \item Использовать  следующие  инструменты:
    \begin{enumerate}
 	      \item \textbf{libccid}: Библиотека  для  работы  со  смарт-картами  и  токенами  в  Linux.
     	  \item \textbf{pcscd}: Фоновая программа  для  работы  с  USB-ридерами  смарт-карт  (необходимо  настроить  максимальное  логирование).
     	  \item \textbf{opensc-tool}: Утилита  командной  строки  для  отправки  APDU-команд  на  токен. 
    \end{enumerate}
    \item Проанализировать  логи  pcscd  и  определить  формат  ccid-пакетов,  которые  отправляются  на  токен. 
    \item Использовать  полученные  данные  для  реализации  взаимодействия  с токеном через WebUSB API в изолированном веб-приложении. 
\end{enumerate} 

\subsection{Запуск Debian на UTM и Windows 11 на Parallels}
Было решено использовать виртуальную машину \href{https://mac.getutm.app}{{UTM}} и с ее помощью запустить \href{https://mac.getutm.app/gallery/debian-12}{{Debian 12}} на macOS. \\
Проблема заключалась в том, что на ARM впринципе невозможно выполнить данное задание по причине того, что поддержка функции \emph{usb-unrestricted} добавлена в Chrome 128, работающий на движке \href{https://www.chromium.org/blink/}{{Blink}}. Но Chrome на macOS, на процессоре Apple Silicon (архитектура ARM), работает на движке WebKit $\Rightarrow$ \emph{usb-unrestricted} на нем не функционирует. 

Так же была предпринята попытка запустить Windows 11 на виртуальной машине \href{https://www.parallels.com}{{Parallels}} для проверки работы проекта WebUSBAuth, но, как позже выяснилось: "В настоящее время невозможно подключить какое-либо USB-устройство к виртуальной машине MacOS Arm." - из \href{https://kb.parallels.com/128867}{{документации}} Parallels. 

\subsection{Запуск проекта на Windows 10 (x64)}
Следующим шагом мной был предпринят запуск проекта на ОС Windows 10 на процессоре x86-64. В отлитчие от запуска в Linux, где необходима установка Google Chrome unstable (версия Chrome для разработчиков), на Windows потребовалось установить Chrome Canary. 

Ниже представлен измененный вариант запуска и установки IWA под Windows.
\begin{tcblisting}{listing engine=minted,minted style=native,
    minted language=python,enhanced,
    colback=terminalColor,colframe=terminalColor,listing only, title=\tikz {
        \node[circle,fill=Button1,inner sep=3pt] (c) at (0,0){};
        \node[circle,fill=Button2,inner sep=3pt] (c) at (0.5,0){};
        \node[circle,fill=Button3,inner sep=3pt] (c) at (1,0){};
    } ~~~~~~Terminal}
    "C:\Users\Филипп\AppData\Local\Google\Chrome SxS\Application\chrome.exe" --enable-features=IsolatedWebApps,IsolatedWebAppDevMode --install-isolated-web-app-from-url=http://localhost:8000
\end{tcblisting}
После установки изолированного приложения Chrome не разрешал работать встроенным JS скриптам внутрь html разметки файла \emph{index.html} и Dev Tools появлялась ошибка \emph{"Refused to execute inline event handler because it violates the following Content Security Policy directive: "script-src 'self'". Either the 'unsafe-inline' keyword..."}. Она решилась путем добавления Content Security Policy в \emph{manifest.json} и в \emph{index.html}, а так же выносом JS скрипта в отдельный файл \emph{main.js}. 

Далее, при попытке соединиться с токеном, нажав на кнопку \emph{"connect reader"}, в консоли возникала ошибка: \emph{"Failed to execute 'open' on
'USBDevice': Access denied".} Как выяснилось, Windows ограничивает доступ к смарт-картам, как к USB-устройству. \\
Это подтверждено в \href{https://groups.google.com/a/chromium.org/g/blink-dev/c/LZXocaeCwDw/m/GLfAffGLAAAJ}{{summary}} описания ограничений доступа к классам устройств из WebUSB. \emph{"These interface classes are already mostly blocked by an operating system’s built-in class drivers."}

\subsection{Успешный запуск проекта на Ubuntu VirtualBox (x64)}
Финальным этапом было решено загрузить бесплатную виртуальную машину VirtualBox на компьютер с процессором x86-64 с дистрибутивом Ubuntu (24.04). \\
Проект успешно запустился и на токен стало возможным отправлять APDU-команды. Был изменен файл \emph{capdu.js}, чтобы при отправке команды в консоль выводился серийный номер токена, по условию задания. Ниже представлена часть, где была изменена команда.  

\lstinputlisting[style=mystyle]{capdu.js}

\end{document}